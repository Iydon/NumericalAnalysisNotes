\usepackage{hdr}
\usepackage{amsmath,ntheorem,amssymb,xfrac,listings,xcolor,enumerate,booktabs}
\usepackage[framemethod=tikz]{mdframed}
\usetikzlibrary{shadows}
% \usepackage{showkeys}
\newmdtheoremenv[%
    outerlinewidth  = 2,%
    roundcorner     = 10pt,%
    backgroundcolor = yellow!40,%
    outerlinecolor  = blue!70!black,%
    innertopmargin  = \topskip,%
    splittopskip    = \topskip,%
    ntheorem        = true,%
    ]{defn}{定义}[section]
\newmdtheoremenv[%
    outerlinewidth  = 1,%
    roundcorner     = 10pt,%
    backgroundcolor = green!40,%
    outerlinecolor  = blue!70!black,%
    innertopmargin  = \topskip,%
    splittopskip    = \topskip,%
    ntheorem        = true,%
    ]{theo}{定理}[section]
% proof
\newenvironment{proof}{{\textbf{[Proof]}\quad}}{\hfill$\square$}
% equation label
% \renewcommand{\theequation}{\arabic{section}.\arabic{equation}}
\numberwithin{equation}{section}
\makeatletter
\@addtoreset{equation}{section}
\makeatother
% section depth
% \setcounter{tocdepth}{2}
% MATLAB
\usepackage{minted}
\newenvironment{matlab}
    {\VerbatimEnvironment\begin{minted}
        [mathescape, linenos, numbersep=5pt,
        gobble=2, %Remove the first n characters from each input line.
        frame=lines, escapeinside=||,
        frame=single, xleftmargin=0em, xrightmargin=0em]{matlab}}
    {\end{minted}}
% Python
\newenvironment{python}
    {\VerbatimEnvironment\begin{minted}
        [mathescape, linenos, numbersep=5pt,
        gobble=2, %Remove the first n characters from each input line.
        frame=lines, escapeinside=||,
        frame=single, xleftmargin=0em, xrightmargin=0em]{python}}
    {\end{minted}}
% command line
\usetikzlibrary{calc}
\definecolor{DarkBlue}{rgb}{.11,.23,.60}
\mdfdefinestyle{commandline}%
{leftmargin=5pt, rightmargin=10pt,innerleftmargin=15pt,
	middlelinecolor=DarkBlue,
	middlelinewidth=2pt,
	frametitlerule=false,
	backgroundcolor=white,%black!10!white,
    roundcorner=3pt,
	frametitle={Command Window},
	frametitlefont={\normalfont\sffamily\color{white}\hspace{-1em}},
	frametitlebackgroundcolor=DarkBlue,
	singleextra={\draw[black!10,line width=12pt]
		($(O)+(7pt,1pt)$) -- ($(O|-P)+(7pt,-\mdfframetitleboxtotalheight)-(0,1pt)$);
		\node[inner sep=0pt,color=black]at ($(O|-P)+(7pt,-25pt)$)%
		{$\scriptstyle f\!x$}; },
	nobreak,
}
\lstnewenvironment{script} {%
	\lstset{language=Matlab,basicstyle=\tiny\ttfamily,breaklines=true,%
		escapeinside = ``,tabsize=2,
        showstringspaces=falsei,
		aboveskip=0pt,belowskip=0pt}}{}
\surroundwithmdframed[style=commandline]{script}
% example
\usepackage{environ}
\usepackage{xparse}
\usetikzlibrary{shapes,decorations}
\definecolor{seco}{RGB}{0,145,215}
\newcommand{\newfancytheoremstyle}[5]{%
    \tikzset{#1/.style={draw=#3, fill=#2,very thick,rectangle, rounded corners, inner sep=10pt, inner ysep=20pt}}
    \tikzset{#1title/.style={fill=#3, text=#2}}
    \expandafter\def\csname #1headstyle\endcsname{#4}
    \expandafter\def\csname #1bodystyle\endcsname{#5}
}
\newfancytheoremstyle{fancythrm}{white!10}{seco}{\bfseries\sffamily}{\sffamily}
\makeatletter
\DeclareDocumentCommand{\newfancytheorem}{ O{\@empty} m m m O{fancythrm} }{%
    % define the counter for the theorem
    \ifx#1\@empty
        \newcounter{#2}
    \else
        \newcounter{#2}[#1]
        \numberwithin{#2}{#1}
    \fi
    %% define the "newthem" environment
    \NewEnviron{#2}[1][{}]{%
        \noindent\centering
        \begin{tikzpicture}
            \node[#5] (box){
                \begin{minipage}{0.93\columnwidth}
                    \csname #5bodystyle\endcsname \BODY~##1
                \end{minipage}};
            \node[#5title, rounded corners, right=10pt] at (box.north west){
                {\csname #5headstyle\endcsname #3 \stepcounter{#2}\csname the#2\endcsname\; ##1}};
            \node[#5title, rounded corners] at (box.east) {#4};
        \end{tikzpicture}
    }[\par\vspace{.5\baselineskip}]
}
% XYPIC
\usepackage[all]{xypic}
\makeatother
\newfancytheorem[section]{example}{例}{$\pi$}
% Algorithmic
\usepackage[linesnumbered,lined,boxed,commentsnumbered]{algorithm2e}
% information
\title{Numerical Analysis}
\author{Iydon}
\date{\today} 