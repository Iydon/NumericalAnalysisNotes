\section{Root-finding problem}
\subsection{The Bisection Method}
\begin{theo}{Intermediate Value Theorem}
$f \in [a,b]$,$\forall k \in f([a,b])$,$\exists c\in [a,b]$,$s.t.\,f(c)=k$。
\end{theo}
\begin{python}
def Bisection(fun, a:float, b:float, max_step:int=128, ...
    eps:float=1e-6)->float:
    mid_last = a
    if fun(a)*fun(b) < 0:
        for i in range(0, max_step):
            mid = (a+b) / 2
            if abs(mid-mid_last)<eps or abs(fun(mid))<eps:
                print("Step: %d\nZero: %fc"%(i, mid))
                return mid
            else:
                if fun(mid)*fun(a)<0:
                    b = mid
                else:
                    a = mid
            mid_last = mid
        print('Bisection cannot be convergent within..
               the pre-set steps.')
\end{python}
\begin{theo}{}
$f\in C[a,b](continuous)$,根据如上算法,$P_i$为mid的序列。如果$\exists$ root P$\in[a,b]$,则有$\left|P_n-P\right|\leq\frac{b-a}{x^n}$。
\end{theo}
\begin{proof}
$\left|b_n-a_n\right|=\frac{b-a}{2^{n-1}}$,
\[
\left|P_n-P\right|\leq\frac{1}{2}(b_n-a_n)\,=\,\frac{b-a}{2^n}
\]
于是$P_n = P + o(2_{-n})$。
\end{proof}

\subsection{Fixed-Point Iteration}
\begin{defn}{Fixed-point Iteration}
对g(P),如果$\forall x\in [a,b]$,如果$\exists$ P $s.t.$ g(P)=P,则称$P$为不动点(fixed point)。\\
如果$g(x)\in C[a,b]$并且$g([a,b])\subset[a,b]$,there exists at least one p$\in[a,b]$,$s.t.$ g(p)=p。
\end{defn}
\begin{theo}{不动点迭代根的存在唯一性定理}
$g(x)\in C[a,b]$,$g([a,b])\subset[a,b]$。$\forall x\in [a,b]$,都有$g'(x)\leq\kappa<1$。
\end{theo}
\begin{proof}\\
    存在性:
    \begin{equation*}
        \begin{cases}
          h(a) =&g(a)-a \geq 0 \\
          h(b) =&g(b)-b \leq 0
        \end{cases}
    \end{equation*}\par
    于是有$h(a)h(b)\leq 0$,则$\exists$ p,$s.t.$ h(p)=0。\\
    唯一性:\par
    假设存在两个根$P_1$,$P_2$,使得$P_1=g(P_1)$,$P_2=g(P_2)$,但是$P_1\neq P_2$。
    \begin{align*}
        \left|g(P_1)-g(P_2)\right| &= \left|g'(\xi)\right|\left| P_1-P_2|\right|,\quad \xi\in [P_1,P_2]. \\
        &\leq \kappa\left|P_1-P_2\right|, contradiction.
    \end{align*}
\end{proof}
\begin{theo}{不动点收敛的充分条件}\label{theo:fixed_point}
g$\in$C[a,b],$g([a,b])\subset[a,b]$,$g'(x)$存在,并且$\left|g'(x)\right|\leq\kappa <1$。$\forall P_0\in[a,b]$,定义序列$P_i=g(P_{i-1}),\quad i=1,2,\cdots$,则$\lim_{n\rightarrow\infty}P_n=P$(P为不动点)。
\end{theo}
\begin{proof}
    \begin{align*}
        \abs{P_n-P} &= \abs{g(P_{n-1}-g(P))} \\
        &= \abs{g'(\xi_{n-1})}\abs{P_{n-1}-P} \\
        &\leq \kappa\abs{P_{n-1}-P} \\
        &\leq \cdots \leq \cdots \\
        &\leq \kappa^n\abs{P_0-P} \rightarrow 0.
    \end{align*}
\end{proof}\par
其中,寻找不动点的代码如下:
\begin{python}
def fixed_point(fun, start:float=0, max_step:int=128, ...
    eps:float=1e-6)->float:
    new_val = fun(start)
    for i in range(0, max_step):
        old_val = new_val
        new_val = fun(old_val)
        if -eps<old_val-new_val<eps:
            print(i)
            return new_val
\end{python}
\begin{theo}{}
如果满足定理\ref{theo:fixed_point},则$p_n$接近$p$的误差可以表示为
\[\left|P_n-P\right|\leq \kappa^n\max\{P_0-a,b-P_0\}\]
并且有
\begin{align*}
    \left|P_n-P\right| &= \left|P_n-P_{n+1}+P_{n+2}-P_{n+3}+\ldots\right| \\
    &\leq \left|P_n-P_{n+1}\right|+\left|P_{n+2}-P_{n+3}\right|+\ldots \\
    &=\kappa^n\left(1+\kappa+\kappa^2+\ldots\right)\left|P_0-P_1\right| \\
    &= \frac{\kappa^n}{1-\kappa}\left|P_0-P_q\right|
\end{align*}
\end{theo}

\subsection{Newton's Method and Its Extensions}
Suppose that $f\in C^2[a,b]$. $p_0\in [a,b]$ and $f'(p_0)\neq 0$ and $\left|p-p_0\right|$ is small.
\[
0 = f(p) = f(p_0)+(p-p_0)f'(p_0)+\frac{(p-p_0)^2}{2}f''(\xi(p)),
\]
$\left|p-p_0\right|$ is small, the term involving $(p-p_0)^2$ is much smaller, then we will get
\[
p \sim p_0 - \frac{f(p_0)}{f'(p_0)}\equiv p_1.
\]
This sets the stage for Newton's method. which starts with an initial approximation $p_0$ and generates the sequence $\{p_n\}_{n=0}^{\infty}$,
\[
p_n = p_{n-1} - \frac{f(p_{n-1})}{f'(p_{n-1})},\quad for\, n\geq 1.
\]
\begin{python}
def Newton_method(f, df, start:float=0.0, max_step:int=32,...
    sign_dig:int=6)->float:
    fun = lambda x: x - f(x)/df(x)
    return fixed_point(fun, start, max_step, sign_dig)

def fixed_point(fun, start:float, max_step:int,...
    sign_dig:int)->float:
    fl = lambda x: round(x, 100)
    eps = 10**(-sign_dig)
    new_val = fun(start)
    for i in range(0, max_step):
        old_val = fl(new_val)
        new_val = fl(fun(old_val))
        if abs(old_val-new_val)<=2*eps:
            return (i, new_val)
    return "Max_step..."
\end{python}
\begin{theo}{牛顿法的收敛性}
Let $f\in C^2[a,b]$. If $p\in (a,b)$ is such that $f(p)=0$ and $f'(p)\neq 0$, then there exists a $\delta>0$ such that Newton's method generates a sequence $\{p_n\}_{n=0}^{\infty}$ converging to $p$ for any initial approximation $p_0\in [p-\delta,p+\delta]$.
\end{theo}
\begin{proof}
证明基于将牛顿迭代法看作 functional iteration scheme $p_n=g(p_{n-1})$,既然$f'(p)\neq 0$,则$\exists\delta_1>0$使得$f'(x)\neq 0$对于所有的$x\in [p-\delta_1,p+\delta_1]$,于是有$g(x)=x-\frac{f(x)}{f'(x)}$。求导后有
\[
g'(x)=1-\frac{f'(x)f'(x)-f(x)f''(x)}{[f'(x)]^2}=\frac{f(x)f''(x)}{[f'(x)]^2}.
\]
于是有$g'(p)=\frac{f(p)f''(p)}{[f'(p)]^2}=0$,又因为$g'$是连续的,$0<k<1$,所以$\exists\delta$,有$0<\delta<\delta_1$,并且$\left|g'(x)\right|\leq k$,对于所有的$x\in [p-\delta,p+\delta]$都成立。\par
接下来需要证明$g$映射$[p-\delta,p+\delta]$到$[p-\delta,p+\delta]$。
\[
\left|g(x)-p\right|=\left|g(x)-g(p)\right|=\left|g'(\xi)\right|\left|x-p\right|\neq k\left|x-p\right|<\left|x-p\right|<\delta.
\]
综上所述,存在$\delta>0$,当$p_0\in [p-\delta,p+\delta]$都有牛顿迭代法收敛到$p$。
\end{proof}

\subsection{The Secant Method}
\[
p_n=p_{n-1}-\frac{f(p_{n-1})(p_{n-1}-p_{n-2})}{f(p_{n-1})-f(p_{n-2})}
\]

\subsection{The Method of False Position}
The method generates approximations in the same manner as the Secant method, but it includes a test to ensure that the root is always bracketed between successive iterations.
\begin{python}
def false_position(f, start:list, max_step:int=32, eps:float=1e-6) -> float:
    """
    False position.
    ----------------------------
    Args:
        f: Function.
        start: List of float, the first iteration point.
        max_step: Integer, max number of iteration.

    Returns:
        Float zero.
        zero | fun(zero)\\sim 0.

    Raises:
        None.
    """
    p = [i    for i in start]
    q = [f(i) for i in start]
    for i in range(max_step):
        _p = p[-1] - q[-1]*(p[-1]-p[0])/(q[-1]-q[0])
        if abs(_p-p[-1]) < eps:
            return (i, _p)
        _q = f(_p)
        if _q*q[-1] < 0:
            p[0] = p[-1]
            q[0] = q[-1]
        p[-1] = _p
        q[-1] = _q
    return False
\end{python}
