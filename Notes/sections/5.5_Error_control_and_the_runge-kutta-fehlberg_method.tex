\section{Error Control and the Runge-Kutta-Fehlberg Method}
\subsection{收敛性与相容性}
\begin{defn}[收敛]
若一种数值方法,对于固定的$x_n=x_0+nh$,当$h\rightarrow 0$时有$y_n\rightarrow y(x_n)$,其中$y(x)$是初值问题的精确解,则称该方法是收敛的
\end{defn}
\begin{theo}[整体截断误差]
假设单步法具有$p$阶精度,且增量函数$\\varphi(x,y,h)$关于$y$满足Lipschitz条件
\[
\abs{\varphi(x,y,h)-\varphi(x,\bar{y},h)}\leq L_\varphi\abs{y-\bar{y}}.
\]
又设初值$y_0$是准确的,即$y_0=y(x_0)$,则其\textbf{整体截断误差}
\[
y(x_n)-y_n\,=\,O(h^p).
\]
\end{theo}
\begin{proof}
设以$\bar{y}_{n+1}$表示取$y_n=y(x_n)$用公式求得的结果,即
\[
\bar{y}_{n+1}=y(x_n)+h\varphi(x_n,y(x_n),h),
\]
则局部截断误差满足,存在常数$C$,使($p$阶精度)
\[
\abs{y(x_{n+1})-\bar{y}_{n+1}}\,\leq\,Ch^{p+1}
\]
所以有
\begin{align*}
    \abs{\bar{y}_{n+1}-y_{n+1}} &\leq \abs{y(x_n)-y_n}+h\abs{\varphi(x_n,y(x_n),h)-\varphi(x_n,y_n,h)} \\
    &\leq (1+hL_\varphi)\abs{y(x_n)-y_n},
\end{align*}
从而有
\begin{align*}
    \abs{y(x_{n+1})-y_{n+1}} &\leq \abs{\bar{y}_{n+1}-y_{n+1}}+\abs{y(x_{n+1})-\bar{y}_{n+1}} \\
    &\leq (1+hL_\varphi)\abs{y(x_n)-y_n} + Ch^{p+1}
\end{align*}
即对整体截断误差$e_n=y(x_n)-y_n$成立下列递推关系
\begin{align*}
    \abs{e_n} &\leq (1+hL_\varphi)\abs{e_{n-1}}+Ch^{p+1} \\
    &\leq (1+hL_\varphi)^n\abs{e_0}+\frac{Ch^p}{L_\varphi}\left[(1+hL_\varphi)^n-1\right]
\end{align*}
再注意到当$x_n-x_0=nh\leq T$时,
\[
(1+hL_\varphi)^n \leq (\mathrm{e}^{hL_\varphi})^n \leq \mathrm{e}^{TL_\varphi}
\]
最终有
\[
\abs{e_n}\leq \abs{e_0}\mathrm{e}^{TL-\varphi}+\frac{Ch^p}{L_\varphi}\left(\mathrm{e}^{TL_\varphi}-1\right)
\]
由此可以断定,如果初值准确,即$e_0=0$,证毕。
\end{proof}
\begin{defn}{相容}
若单步法的增量函数$\varphi$满足$\varphi(x,y,0)=f(x,y)$,则称单步法与初值问题是相容的。
\end{defn}
\begin{defn}{稳定}若一种数值方法在节点值$y_n$上大小为$\delta$的扰动,于以后各节点值$y_m(m>n)$上产生的偏差不超过$\delta$,则称该方法是\textbf{稳定的}。
\end{defn}
为了只考虑数值方法本身,通常只检验将数值方法用于解模型方程的稳定性,模型方程为
\[
y'=\lambda y.
\]
其中$\lambda$为复数,这个方程分析简单,对一般方程可以通过局部线性优化转化为这种形式,例如在$\bar{x},\bar{y}$的邻域,可展开为
\[
y'=f(x,y)=f(\bar{x},\bar{y})+f_x'(\bar{x},\bar{y})(x-\bar{x})+f_y'(\bar{x},\bar{y})(y-\bar{y})+\cdots
\]
\begin{defn}
单步法对于解模型方程,若得到的解$y_{n+1}=E(h\lambda)y_n$,满足$\abs{E(h\lambda)}<1$,则称该单步法是\empty{绝对稳定的},在$\mu=h\lambda$的平面上,使$\abs{E(h\lambda)}<1$的变量围成的区域,称为\textbf{绝对稳定域},它与实轴的交称为\textbf{绝对稳定区间}。
\end{defn}

\begin{tabular}{ll}
欧拉法 & $E(h\lambda)=1+h\lambda$ \\
二阶R-K方法 & $E(h\lambda)=1+h\lambda+\frac{(h\lambda)^2}{2}$ \\
三阶R-K方法 & $E(h\lambda)=1+h\lambda+\frac{(h\lambda)^2}{2}+\frac{(h\lambda)^3}{6}$ \\
四阶R-K方法 & $E(h\lambda)=1+h\lambda+\frac{(h\lambda)^2}{2}+\frac{(h\lambda)^3}{6}+\frac{(h\lambda)^4}{24}$ \\
后退欧拉法 & $E(h\lambda)=\frac{1}{1-h\lambda}$ \\
梯形法 & $E(h\lambda)=\frac{2+h\lambda}{2-h\lambda}$
\end{tabular} 